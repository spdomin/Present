%%%%%%%%%%%%%%%%%%%%%%%%%%%%%%%%%%%%%%%%%
% Twenty Seconds Resume/CV
% LaTeX Template
% Version 1.1 (8/1/17)
%
% This template has been downloaded from:
% http://www.LaTeXTemplates.com
%
% Original author:
% Carmine Spagnuolo (cspagnuolo@unisa.it) with major modifications by 
% Vel (vel@LaTeXTemplates.com)
%
% License:
% The MIT License (see included LICENSE file)
%
% Modified by:
% Stefan P. Domino with retention of the MIT LICENSE
%
%%%%%%%%%%%%%%%%%%%%%%%%%%%%%%%%%%%%%%%%%

%----------------------------------------------------------------------------------------
%	PACKAGES AND OTHER DOCUMENT CONFIGURATIONS
%----------------------------------------------------------------------------------------

\documentclass[letterpaper]{twentysecondcv_spd} % a4paper for A4

%----------------------------------------------------------------------------------------
%	 PERSONAL INFORMATION
%----------------------------------------------------------------------------------------

% If you don't need one or more of the below, just remove the content leaving the command, e.g. \cvnumberphone{}

\profilepic{cvMontageSmall.png} % Profile picture

\cvname{Stefan P. Domino Ph.D.} % Your name
\cvjobtitle{Computational Scientist} % Job title/career

\cvaddress{Sandia National Laboratories \newline 1515 Eubank Blvd SE \newline Albuquerque, NM, 87123} % Short address/location, use \newline if more than 1 line is required
\cvsite{ \underline{https://github.com/NaluCFD}} % germane website
\cvmail{spdomin@sandia.gov \newline spdomin@stanford.edu \newline spdomino@comeri.org \newline spdomino@gmail.com} % Email address

%----------------------------------------------------------------------------------------

\begin{document}

\pagestyle{headings}
\setcounter{page}{1}
\pagenumbering{roman}

%----------------------------------------------------------------------------------------
%	 What I do
%----------------------------------------------------------------------------------------

\aboutme{ I transform people's understanding of the world by deploying high-performing computational 
fluid dynamics tools. While some draw analogy to such computational tools as the rasp in Michelangelo's hand, 
I choose to view the partnership analogous to that of the luthier and the violinist in that both are required 
to make new that which was formerly unheard.} 

%----------------------------------------------------------------------------------------
%	 SKILLS
%----------------------------------------------------------------------------------------

% Skill bar section, each skill must have a value between 0 an 6 (float)
\skills{{Passionate/5.45}, {Driven/5.5}, {Outgoing/4.0}, {Software Development/5.0}, {Next-generation-platforms/4.8}, {Turbulence/5.1}, {low-Mach Fluids*/5.4}}

%------------------------------------------------

%----------------------------------------------------------------------------------------

\makeprofile % Print the sidebar

%----------------------------------------------------------------------------------------
%	 INTERESTS
%----------------------------------------------------------------------------------------

\section{Interests}

My professional interest resides in the development and deployment of computational fluid dynamics (CFD)
tools to facilitate a transformative understanding 
of otherwise intractable physical phenomena. By exercising these tools, in partnership with theory and
experiments, a window into complex coupled processes can be illuminated. Studying low-Mach multi-physics 
applications that include turbulence, variable-density effects, buoyancy, multiphase, and chemical reactions often times
reveals extraordinarily complex fluids, thermal, and species structures thereby allowing that which is
generally unseen to be fully appreciated. Fostering partnerships within a diverse and high-performing 
team to solve grand-challenge problems provides me with ample motivation to work in the complex field of CFD.


%----------------------------------------------------------------------------------------
%	 Current Posiitons
%----------------------------------------------------------------------------------------

\section{Current Position(s) Held}

\begin{twentyshort} % Environment for a short list with no descriptions
	\twentyitemshort{Sandia}{Computational Scientist; Engineering Science (1500).}
  	\twentyitemshort{Stanford} {Adjunct Professor; School of Engineering, Institute for Computational and Mathematical Engineering (ICME), \underline{https://icme.stanford.edu}.}
  	\twentyitemshort{COMERI}{CEO/President/Senior Technical Scientist;  \underline{https://comeri.org}.}
  \end{twentyshort}
  
%----------------------------------------------------------------------------------------
%	 EDUCATION
%----------------------------------------------------------------------------------------

\section{Education}

\begin{twenty} % Environment for a list with descriptions
	\twentyitem{1999}{Doctor of Philosophy} {\normalfont Chemical Engineering, University of Utah} {Researched and deployed advanced
modeling and simulation techniques to more accurately predict the oxides of nitrogen ($NO_x$) in multiphase combustion applications. Advisor: Professor Philip Smith}
	\twentyitem{1995}{First Year Graduate Classes} {\normalfont Chemical Engineering, University of Washington} {Researched atomic force microscopy applied to
          measuring DNA base pair hydrogen bonding. Advisor: Professor Buddy Ratner.}
	\twentyitem{1994}{Bachelor of Science} {\normalfont Chemical Engineering, University of Utah} {Researched the use of per-fluorocarbons for advanced mammalian bioreactor design. Advisor: Professor Edward Trujillo.}
\end{twenty}

%----------------------------------------------------------------------------------------
%	 PUBLICATIONS
%----------------------------------------------------------------------------------------

\section{Recent Peer-reviewed Publications}

\begin{twentyshort} % Environment for a short list with no descriptions

\twentyitemshort{2021} {Domino, S. P., Hewson, J., Knaus, R., Hansen, M., \textit{Predicting large-scale pool fire dynamics using an unsteady flamelet- and large-eddy simulation-based model suite}, Phys. Fluids, https://doi.org/10.1063/5.0060267 (Editor's pick).}

\twentyitemshort{2021} {Domino, S. P.,  \textit{A case study on pathogen transport, deposition, evaporation and transmission: linking high-fidelity computational fluid dynamics simulations to probability of infection}, Int. J. CFD, https://doi.org/10.1080/10618562.2021.1905801.}

\twentyitemshort{2021} {Domino, S. P., Pierce, F., Hubbard, J.,  \textit{A multi-physics computational investigation of droplet pathogen transport emanating from synthetic coughs and breathing}, Atom. Sprays, https://doi.org/10.1615/AtomizSpr.2021036313.}

\twentyitemshort{2019} {Jofre, L., Domino, S. P., Iaacarino, G.,  \textit{Eigensensitivity analysis of subgrid-scale stresses in large-eddy simulation of a turbulent axisymmetric jet}, Int. J. Heat Mass, https://doi.org/DOI:10.1016/J.IJHEATFLUIDFLOW.2019.04.014.}

\twentyitemshort{2019} {Domino, S. P., Sakievich, P., Barone, M.,  \textit{An assessment of atypical mesh topologies for low-Mach large-eddy simulation}, Comp. Fluids, https://doi.org/10.1016/j.compfluid.2018.12.002.}

\twentyitemshort{2018} {Domino, S. P.,  \textit{Design-order, non-conformal low-Mach fluid algorithms using a hybrid CVFEM/DG approach}, J. Comput. Phys., https://doi.org/10.1016/j.jcp.2018.01.007.}

\twentyitemshort{2018} {Jofre, L., Domino, S. P., Iaacarino, G.,  \textit{A framework for characterizing structural uncertainty in large-eddy simulation closures}, Flow Turb. Combust., https://doi.org/10.1007/s10494-017-9844-8.}

\end{twentyshort} 

%----------------------------------------------------------------------------------------
%	 SECOND PAGE EXAMPLE
%----------------------------------------------------------------------------------------

\newpage % Start a new page

%----------------------------------------------------------------------------------------
%	 talk about my goals
%----------------------------------------------------------------------------------------

\mygoals{My primary career goal centers on extending state-of-the art in CFD methods 
to facilitate the advanced deployment of credible tools that support a wide range 
of atypical, e.g., fire, wind/wave-energy, computational ethology, etc., multi-physics applications. Mentoring, 
teaching, and motivating the next generation of computational scientists captures my core passion. } 

% investment areas, value between 0 an 6 (float)
\investment{{Accidental Fires/5.8}, {Wildfires/5.9}, {VVUQ/5.7}, {Marine Ethology/5.95}, {Ember Transport/5.2}, {Next-Generation Platforms/5.54}, {Wave Energy*/5.8}}


\makeprofileSecond % Print the sidebar

%----------------------------------------------------------------------------------------
%	 EXPERIENCE
%----------------------------------------------------------------------------------------

\section{Recent Experience}

\begin{twenty} % Environment for a list with descriptions
   \twentyitem{2005-now}{Principal Member of the Technical Staff} {Sandia National Laboratories} {My experience at Sandia rests within low-Mach 
    turbulent fluid mechanics methods development for complex systems that drive the coupling of mass, momentum, species and energy transport. 
    As PI, my research projects reside within the intersection of physics model development, numerical methods research, V\&V techniques exploration, and 
    high-performance computing and coding methods for low-Mach turbulent flow. I am the originator of the BSD open-source Nalu
    code base, \underline{https://github.com/NaluCFD}. In my role as a technical staff, I am proud to have served the Lab's response to National crises 
    such as Deep Water Horizon and the COVID-19 pandemic.}
   
   \twentyitem{2020-now}{Adjunct Professor} {Stanford/ICME} {Co-teaching responsibilities for Stanford's ME469 Mechanical Engineering graduate CFD 
   class where Nalu is used as pedagogical tool to bridge foundational numerical methods development and practical production CFD. I also support the 
   mentoring of graduate students and post-doctoral candidates.}

   \twentyitem{2018-now}{CEO/President/Senior Technical Staff} {COMERI} {Lead the management, research, and funding objectives
    for the Computational Marine Ethology Research Institute (COMERI) - a 501(c)(3) nonprofit research Institute that drives 
    foundational understanding of marine ethology using first-principles physics.}
    
  \twentyitem{2001-2005}{Senior Member of the Technical Staff} {Sandia National Laboratories}  {PI
    and lead developer for the generally unstructured, massively parallel Sierra/Fuego code base and team contributor to the NNSA Defense Programs Awards of Excellence for 
    significant contributions Stockpile Stewardship Program.}
    
  \twentyitem{2000-2001}{Postdoctoral appointee} {Sandia National Laboratories}  {Development of a smoke transport simulation tool for cargo bay 
  fires in support of the FAA's response to ValueJet Flight 592. This work was recognized as part of the NASA Associate Administrator's Choice Award for Outstanding Accomplishment, (Glenn Research Center) 
  and a R\&D 100 Award for the development of a multi-parameter, micro-sensor-based low false alarm fire detection system.}
  
\end{twenty}



%----------------------------------------------------------------------------------------
%	 Projects lead
%----------------------------------------------------------------------------------------

\section{Noteable Projects as PI}

\begin{twentyshort} % Environment for a short list with no descriptions
  \twentyitemshort{2019-now}{\textit{Uncertainty quantification in crash-and-burn environments}. Exploring fire dynamics for
    accident scenarios that include varying pool shape and crosswind magnitude; developing structural uncertainties for large-eddy simulation through
    eigenvalue decomposition and perturbation of stresses towards limiting turbulence states. }
  \twentyitemshort{2020-now}{\textit{Agile Physics and Engineering Models}. Developing advanced coupling techniques for thermal response in the
    presence of thermal radiation; elucidation of non-isothermal jet impingement physics; wall-resolved large-eddy simulation modeling.}
  \twentyitemshort{2021-now}{\textit{Developing credible high-fidelity mod/sim tools for wave energy converter design}. Development of implicit
    overset methods coupled to six-DOF and volume of fluid transport. }
  \twentyitemshort{2020-2021}{\textit{COVID-19 Transportation and Transmission}. High-fidelity pathogen modeling approaches for breathing and coughing events
    using an Eulerian/point-Lagrangian multi-physics paradigm that allows for the ability to distinguish between droplets that deposit and
    those that form persisting aerosol plumes.}
  \twentyitemshort{2003 - 2006}{\textit{Sierra/Fuego Integrated Codes Project}. Fuego is the Sandia National Laboratories turbulent reacting flagship
    fire physics simulation tool that supports Science-based Stockpile Stewardship.}
\end{twentyshort}

%----------------------------------------------------------------------------------------
%	 Third PAGE EXAMPLE
%----------------------------------------------------------------------------------------

\newpage % Start a new page

%----------------------------------------------------------------------------------------
%	 talk about why I do what I do
%----------------------------------------------------------------------------------------

\whyidoit{ The ability to explore multi-physics applications from a foundational modeling
and simulation perspective is critical to future scientific advances. This high-level motivation
has driven my desire to work within the intersection of physics elucidation, numerical 
methods development, and code development. More recently, the ability to deploy advanced uncertainty
quantification (UQ) techniques to drive physics understanding, which may include structural uncertainty methods, 
machine learning approaches, etc., has transformed the former research paradigm. } 

%----------------------------------------------------------------------------------------
%	 favorite things
%----------------------------------------------------------------------------------------

% favorite things bar section, value between 0 an 6 (float)
\favthings{{Ocean/5.9}, {Science/5.7}, {Pursuit of Knowledge/5.6}, {CFD/5.8}, {Snow/5.2}, {Mountains/5.8}, {Family/6.0}, {Replication of Past Work*/2.0}}

\makeprofileThird % Print the sidebar


%----------------------------------------------------------------------------------------
%	 Select Publications
%----------------------------------------------------------------------------------------

\section{Noteworthy Publications/Book Chapters and Presentations}

\begin{twentyshort} % Environment for a short list with no descriptions
  \twentyitemshort{2017}{Eldred, M., Ng, A., Barone, M., Domino, S. P.,
    \textit{Multifidelity uncertainty quantification using spectral stochastic discrepancy models}; In:
    Ghanem R., Higdon D., Owhadi H. (eds) Handbook of Uncertainty Quantification.}
  \twentyitemshort{2014}{Lin, P., Bettencourt, M., Domino, S. P., et al.,
    \textit{Towards extreme-scale simulations for low-Mach fluids with second-generation Trilinos},
    Parallel Processing Letters, 24 (4).}
  \twentyitemshort{2013}{Domino, S. P., \textit{A reflection of recent ASC milestones in support of the abnormal/thermal environment} 
    Sandia National Laboratories Technical Report, SAND2013-3927P.}
  \twentyitemshort{2009}{Domino, S. P., \textit{Computational approaches to multi-physics applicaitons: Predicting an object's thermal
    resposne within a turbulent reacting, participating media radiation enviornment.} 
     Plenary Invitation, SIAM Conference on Computational Science and Engineering.}
\end{twentyshort}


%----------------------------------------------------------------------------------------
%	 AWARDS
%----------------------------------------------------------------------------------------

\section{Distinguished Awards}

\begin{twentyshort} % Environment for a short list with no descriptions
  \twentyitemshort{2017}{Sheldon R. Tieszen Sandia National Laboratories Engineering Sciences Award for a
    distinguished career in pursuit of technical excellence.}
\end{twentyshort}

%----------------------------------------------------------------------------------------
%	 Noteworth Experiences
%----------------------------------------------------------------------------------------

\section{Noteworthy Experiences}

\begin{twentyshort} % Environment for a short list with no descriptions
	\twentyitemshort{2020 \& 2021}{Co-teaching Me469, Stanford Mechanical Engineering Departments Graduate Introduction to CFD class.}
	\twentyitemshort{2006-2018}{Six-time visiting scholar at Stanford's Center for Turbulence Research.}
	\twentyitemshort{2000-now}{Numerous internal and external peer-reviews supported including journals, DOE panels, NSF, and others.}
	\twentyitemshort{2000-now}{Mentoring of four post-doctoral researchers and five graduate students.}
\end{twentyshort}

\section{References}

\begin{twentysingle} % Environment for a short list with no descriptions
  \twentyitemsingle{Please contact me for a comprehensive list of references.}
\end{twentysingle}

\section{Review}

Dr. Stefan Domino is a computational domain specialist researcher who develops tools and techniques
to support advancement of multi-physics understanding of complex phenomena including turbulent 
fluid mechanics, heat transfer, and chemical reactions.

\end{document}
