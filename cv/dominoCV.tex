%%%%%%%%%%%%%%%%%%%%%%%%%%%%%%%%%%%%%%%%%
% Twenty Seconds Resume/CV
% LaTeX Template
% Version 1.1 (8/1/17)
%
% This template has been downloaded from:
% http://www.LaTeXTemplates.com
%
% Original author:
% Carmine Spagnuolo (cspagnuolo@unisa.it) with major modifications by 
% Vel (vel@LaTeXTemplates.com)
%
% License:
% The MIT License (see included LICENSE file)
%
% Modified by:
% Stefan P. Domino with retention of the MIT LICENSE
%
%%%%%%%%%%%%%%%%%%%%%%%%%%%%%%%%%%%%%%%%%

%----------------------------------------------------------------------------------------
%	PACKAGES AND OTHER DOCUMENT CONFIGURATIONS
%----------------------------------------------------------------------------------------

\documentclass[letterpaper]{twentysecondcv_spd} % a4paper for A4

%----------------------------------------------------------------------------------------
%	 PERSONAL INFORMATION
%----------------------------------------------------------------------------------------

% If you don't need one or more of the below, just remove the content leaving the command, e.g. \cvnumberphone{}

\profilepic{cvMontageSmall.png} % Profile picture

\cvname{Stefan P. Domino Ph.D.} % Your name

\cvjobtitle{Computational Scientist} % Job title/career

\cvaddress{Sandia National Laboratories \newline 1515 Eubank Blvd SE \newline Albuquerque, NM, 87123} % Short address/location, use \newline if more than 1 line is required
\cvmail{spdomin@sandia.gov \newline spdomin@stanford.edu } % Email address
\cvsite{ \underline{\normalsize{github.com/NaluCFD}}} % germane website

%----------------------------------------------------------------------------------------

\begin{document}

\pagestyle{headings}
\setcounter{page}{1}
\pagenumbering{roman}

%----------------------------------------------------------------------------------------
%	 What I do
%----------------------------------------------------------------------------------------
\aboutme{ I transform people's understanding of the world by deploying high-performing computational fluid dynamics tools. While some draw analogy to such computational tools as the rasp in Michelangelo's hand, I choose to view the partnership analogous to that of the luthier and the violinist in that both are required to make new that which was formerly unheard.} 

%----------------------------------------------------------------------------------------
%	 SKILLS
%----------------------------------------------------------------------------------------
% Skill bar section, each skill must have a value between 0 an 6 (float)
\skills{{Passionate/5.45}, {Driven/5.5}, {Extrovert/4.0}, {Software Development/5.0}, {Next-generation-platforms/4.8}, {Turbulence/5.1}, {low-Mach Fluids/5.4}}

\makeprofile % Print the sidebar

%----------------------------------------------------------------------------------------
%	 Interests
%----------------------------------------------------------------------------------------
\section{Interests}

My professional interest resides in the development and deployment of computational fluid dynamics (CFD) tools to facilitate a transformative understanding of otherwise intractable physical phenomena. By exercising these tools, in partnership with theory and experiments, a window into complex coupled processes can be illuminated. Studying low-Mach multi-physics applications that include turbulence, variable-density effects, buoyancy, multiphase, and chemical reactions often times reveals extraordinarily complex fluids, thermal, and species structures thereby allowing that which is generally unseen to be fully appreciated. Fostering partnerships within a diverse and high-performing team to solve grand-challenge problems provides me with ample motivation to work in the complex field of CFD.

%----------------------------------------------------------------------------------------
%	 Current Position(s) Held
%----------------------------------------------------------------------------------------
\section{Current Position(s) Held}

\begin{twentyshort}

	\twentyitemshort{Sandia}{Computational Scientist; Center: Engineering Sciences, Group: Computational Thermal and Fluid Mechanics (1541); Distinguished Member of the Technical Staff.}
	
	\twentyitemshort{COMERI} {CEO/President/Esteemed Technical Staff, Computational Marine Ethology Research Institute, \underline{https://comeri.org}.}

  	\twentyitemshort{Stanford} {Adjunct Professor; School of Engineering, Institute for Computational and Mathematical Engineering (ICME), \underline{https://icme.stanford.edu}.}
		
  \end{twentyshort}
  
%----------------------------------------------------------------------------------------
%	 Education
%----------------------------------------------------------------------------------------
\section{Education}

\begin{twenty}
	\twentyitem{1999}{Doctor of Philosophy} {\normalfont Chemical Engineering, University of Utah} {Researched and deployed advanced
modeling and simulation techniques to more accurately predict the oxides of nitrogen ($NO_x$) in multiphase combustion applications. Advisor: Professor Philip Smith}
%	\twentyitem{1995}{First Year Graduate Classes} {\normalfont Chemical Engineering, University of Washington} {Researched atomic force microscopy applied to
%          measuring DNA base pair hydrogen bonding. Advisor: Professor Buddy Ratner.}
	\twentyitem{1994}{Bachelor of Science} {\normalfont Chemical Engineering, University of Utah} {Researched the use of per-fluorocarbons for advanced mammalian bioreactor design. Advisor: Professor Edward Trujillo.}
	
\end{twenty}

%----------------------------------------------------------------------------------------
%	Recent Experience
%----------------------------------------------------------------------------------------
\section{Recent Experience}

\begin{twenty}

	\twentyitem{2001-now}{Member of the Technical Staff} {Sandia National Laboratories} {Extreme scale, low-Mach turbulent fluid mechanics methods development for complex systems that drive the coupling of mass, momentum, species and energy transport. As PI, my research resides within the intersection of physics model development, numerical methods research, V\&V techniques exploration, and high-performance computing and coding methods for low-Mach flow; originator of the open-source Nalu code base, \underline{https://github.com/NaluCFD}. I am proud to have served the Lab's response to National crises such as Deep Water Horizon and the COVID-19 pandemic. Served as PI and lead developer for the generally unstructured, massively parallel Sierra/Fuego code base and was a team contributor to the NNSA Defense Programs Awards of Excellence for significant contributions Stockpile Stewardship Program. Promotion to Senior Member of the Technical Staff (2001), Principal Member of the Technical Staff (2005), and Distinguished Member of the Technical Staff (2022).}
	
	\twentyitem{2018-now}{CEO/President/Esteemed Technical Staff} {COMERI} {Lead management, research, and funding objectives for the Computational Marine Ethology Research Institute (COMERI) - a 501(c)(3) nonprofit research Institute that drives foundational understanding of marine ethology using first-principles physics.}
	
	\twentyitem{2021-now}{Adjunct Professor} {Stanford/ICME} {Teaching responsibilities for Stanford's ME469 Mechanical Engineering graduate  \textit{Computational Methods for Fluid Dynamics} class where Nalu is used as pedagogical tool to bridge foundational numerical methods development and practical production CFD; support mentoring of graduate students and post-doctoral appointees.}
	
\end{twenty}

%----------------------------------------------------------------------------------------
%	 SECOND PAGE EXAMPLE
%----------------------------------------------------------------------------------------

\newpage % Start a new page

%----------------------------------------------------------------------------------------
%	 talk about my goals
%----------------------------------------------------------------------------------------

\mygoals{My primary career goal centers on extending state-of-the art in CFD methods to facilitate the advanced deployment of credible tools that support a wide range of atypical, e.g., fire, wind and wave-energy, computational ethology, with an emphasis on multi-physics applications. Mentoring, teaching, and motivating the next generation of computational scientists captures my core passion. } 

% investment areas, value between 0 an 6 (float)
\investment{{Accidental-Wildfire Fires/5.9}, {CDR-DAC/5.9}, {VVUQ/5.7}, {Marine Ethology/5.95}, {Ember Transport/5.2}, {Next-Generation Platforms/5.54}, {Wave Energy/5.8}}

\makeprofileSecond % Print the sidebar

%----------------------------------------------------------------------------------------
%	 Recent Peer-reviewed Publications
%----------------------------------------------------------------------------------------
\section{Recent Peer-reviewed Publications}

\begin{twentyshort}

	\twentyitemshort{2024}{Domino, S. P. \textit{On the subject of large-scale pool fires and turbulent boundary layer interactions}, Phys. Fluids, https://doi.org/10.1063/5.0196265.}

	\twentyitemshort{2023}{Domino, S. P., Wenzel, E., \textit{A direct numerical simulation study for confined non-isothermal jet impingement at moderate nozzle-to-plate distances: capturing jet-to-ambient density effects}, Int. J. Heat Mass Trans, https://doi.org/10.1016/j.ijheatmasstransfer.2023.124168.}

	\twentyitemshort{2022}{Scott, S. N., Domino, S. P., \textit{A computational examination of large-scale pool fires: variations in crosswind velocity and pool shape}, Flow, https://doi.org/10.1017/flo.2022.26.}
	
	\twentyitemshort{2022}{Domino, S. P., Horne, W., \textit{Development and deployment of a credible unstructured, six-DOF, implicit low-Mach overset  
simulation tool for wave energy applications}, Renewable Energy, https://doi.org/10.1016/j.renene.2022.09.005.}

	\twentyitemshort{2021} {Domino, S. P., Hewson, J., Knaus, R., Hansen, M., \textit{Predicting large-scale pool fire dynamics using an unsteady flamelet- and large-eddy simulation-based model suite}, Phys. Fluids, https://doi.org/10.1063/5.0060267 (Editor's pick).}
	
	\twentyitemshort{2021} {Domino,  S. P., \textit{A case study on pathogen transport, deposition, evaporation and transmission: linking high-fidelity computational fluid dynamics simulations to probability of infection}, Int. J. CFD, https://doi.org/10.1080/10618562.2021.1905801.}
	
	\twentyitemshort{2021} {Domino, S. P., Pierce, F., Hubbard, J.,  \textit{A multi-physics computational investigation of droplet pathogen transport emanating from synthetic coughs and breathing}, Atom. Sprays, https://doi.org/10.1615/AtomizSpr.2021036313.}
	
	\twentyitemshort{2019} {Jofre, L., Domino, S. P., Iaacarino, G.,  \textit{Eigensensitivity analysis of subgrid-scale stresses in large-eddy simulation of a turbulent axisymmetric jet}, Int. J. Heat Fluid Flow, https://doi.org/DOI:10.1016/J.IJHEATFLUIDFLOW.2019.04.014.}
	
	\twentyitemshort{2019} {Domino, S. P., Sakievich, P., Barone, M., \textit{An assessment of atypical mesh topologies for low-Mach large-eddy simulation}, Comp. Fluids, https://doi.org/10.1016/j.compfluid.2018.12.002.}
	
	\twentyitemshort{2018} {Domino, S. P., \textit{Design-order, non-conformal low-Mach fluid algorithms using a hybrid CVFEM/DG approach}, J. Comput. Phys., https://doi.org/10.1016/j.jcp.2018.01.007.}
	
	\twentyitemshort{2018} {Jofre, L., Domino, S. P.,  Iaacarino, G., \textit{A framework for characterizing structural uncertainty in large-eddy simulation closures}, Flow Turb. Combust., https://doi.org/10.1007/s10494-017-9844-8.}

\end{twentyshort} 

%----------------------------------------------------------------------------------------
%	 Noteworthy Publications
%----------------------------------------------------------------------------------------
\section{Noteworthy Publications}

\begin{twentyshort}

	\twentyitemshort{2023}{Benjamin, M., Domino, S. P., Iaccarino, G., \textit{Neural networks for large eddy simulations of wall-bounded turbulence: numerical experiments and challenges}, Eur. Phys. J. E, https://doi.org/10.1140/epje/s10189-023-00314-6.}

	\twentyitemshort{2022}{Hubbard, J., Hansen, M., Kirsch, J., Hewson, J., Domino, S. P., \textit{Medium scale methanol pool fire model validation}, J. Heat Transfer, https://doi.org/10.1115/1.4054204.}
	
	\twentyitemshort{2022} {Barone, M., Ray, J., Domino, S. P., \textit{Feature selection, clustering, and prototype placement for turbulence datasets"}, AIAA J., https://doi.org/10.2514/1.J060919.}
	
	\twentyitemshort{2014}{Lin, P., Bettencourt, M., Domino, S. P., et al., \textit{Towards extreme-scale simulations for low-Mach fluids with second-generation Trilinos}, Parallel Processing Letters, https://doi.org/10.1142/S0129626414420055.}
	
\end{twentyshort}

%----------------------------------------------------------------------------------------
%	 Under-Review/In-Preparation Manuscripts
%----------------------------------------------------------------------------------------
%\section{Under-Review/In-Preparation Manuscripts}
%
%\begin{twentyshort}
%
%	\twentyitemshort{2022}{Domino, S. P., Wenzel, E., \textit{A Direct Numerical Simulation study for Re$_\tau$~505 non-isothermal jet impingement}, in prep.}
%
%\end{twentyshort}

%----------------------------------------------------------------------------------------
%	 Third PAGE EXAMPLE
%----------------------------------------------------------------------------------------

\newpage % Start a new page

%----------------------------------------------------------------------------------------
%	 talk about why I do what I do
%----------------------------------------------------------------------------------------

\whyidoit{ The ability to explore multi-physics applications from a foundational modeling and simulation perspective is critical to future scientific advances. This high-level motivation has driven my desire to work within the intersection of physics elucidation, numerical methods development, and code development. More recently, the ability to deploy advanced uncertainty quantification (UQ) techniques to drive physics understanding, which may include structural uncertainty methods, machine learning approaches, etc., has transformed the former research paradigm. } 

%----------------------------------------------------------------------------------------
%	 favorite things
%----------------------------------------------------------------------------------------

% favorite things bar section, value between 0 an 6 (float)
\favthings{{Ocean/5.9}, {Science/5.7}, {Pursuit of Knowledge/5.6}, {CFD/5.8}, {Snow/5.2}, {Mountains/5.8}, {Family/6.0}, {Replication of Past Work/2.0}}

\makeprofileThird % Print the sidebar

%----------------------------------------------------------------------------------------
%	 Notable Projects as PI
%----------------------------------------------------------------------------------------
\section{Notable Projects as PI}

\begin{twentyshort} 

	\twentyitemshort{2020-now}{\textit{Agile Physics and Engineering Models}. Advanced thermal/thermal radiation coupling techniques; elucidation of non-isothermal jet impingement physics; wall-resolved large-eddy simulation modeling, data science turbulence modeling.}
	
	\twentyitemshort{2019-now}{\textit{VVUQ Methods for Turbulent Flow}. Exploring fire dynamics for accident scenarios that include varying pool shape and crosswind magnitude; developing structural uncertainties for large-eddy simulation through eigenvalue decomposition and perturbation of stresses towards limiting turbulence states.}
	
	\twentyitemshort{2021-2022}{\textit{Developing credible high-fidelity mod/sim tools for wave energy converter design}. Development of implicit overset methods coupled to six-DOF and volume of fluid transport.}
	
	\twentyitemshort{2020-2021}{\textit{COVID-19 Transportation and Transmission}. High-fidelity pathogen modeling approaches for breathing and coughing events using an Eulerian/point-Lagrangian multi-physics paradigm that allows for the ability to distinguish between droplets that deposit and those that form persisting aerosol plumes.}
	
	\twentyitemshort{2012-2015}{\textit{Computer Science Advanced Research: Core Computational Methodologies}. Portfolio manager for a multi-discipline Advanced Simulation and Computing (Research Foundations) included funding decisions (\$1.25 million per year) and technical oversight of advanced methods of algebraic multigrid, the development of Helmholtz solvers, etc.}
	
	\twentyitemshort{2003 - 2006}{\textit{Sierra/Fuego Integrated Codes Project}. Fuego is the Sandia National Laboratories turbulent reacting flagship fire physics simulation tool that supports Science-based Stockpile Stewardship.}
	
\end{twentyshort}

%----------------------------------------------------------------------------------------
%	 Distinguished Awards
%----------------------------------------------------------------------------------------

\section{Distinguished Awards}

\begin{twentyshort}

	\twentyitemshort{2017}{Sheldon R. Tieszen Sandia National Laboratories Engineering Sciences Award for a distinguished career in pursuit of technical excellence.}
	
\end{twentyshort}

%----------------------------------------------------------------------------------------
%	 Noteworthy Presentations
%----------------------------------------------------------------------------------------
\section{Noteworthy Presentations}

\begin{twentyshort}

	\twentyitemshort{2023}{Domino, S. P., \textit{Building a Credible, Open-Source High-Fidelity Computational Fluid Dynamics Tool Suitable for Renewable Wave Energy Applications}, Invited Stanford CTR Tea Seminar Series.}

	\twentyitemshort{2022}{Domino, S. P., \textit{Exploring high-fidelity computational fluid dynamics approaches for airborne pandemic risk mitigation}, Invited Stanford CTR Tea Seminar Series.}
	
	\twentyitemshort{2021}{Domino, S. P., \textit{A historical perspective on Sandia National Laboratories fire science mod/sim philosophy: The role of high performance computing and unstructured numerical methods advances}, Invited University of Utah Graduate Seminar Series.}
	
	\twentyitemshort{2020}{Domino, S. P., \textit{An evolution of a mindset:  A historical perspective on Sandia National Laboratories Fire Science philosophy}, Invited Stanford CTR Tea Seminar Series.}
	
	\twentyitemshort{2018}{Domino, S. P., \textit{Multi-phase use cases within the abnormal thermal environment}, Invited PSAAP-2 Multi-phase Workshop.}
	
	\twentyitemshort{2018}{Domino, S. P., \textit{The suitability of hybrid meshes for low-Mach large-eddy simulation}, Invited LLNL/CASC Seminar Series.}
	
	\twentyitemshort{2017}{Domino, S. P., \textit{ECP ExaWind experience in transitioning Nalu to MPI+x}, Invited PSAAP-2 Review.}
	
%	\twentyitemshort{2014}{Domino, S. P., \textit{Leveraging the ASC Sierra Mechanics tools for battery fire prediction}, DOE Office of Electricity Energy Storage Peer Review.}
	
	\twentyitemshort{2009}{Domino, S. P., \textit{Computational approaches to multi-physics applications: Predicting an object's 
thermal response within a turbulent reacting, participating media radiation environment}, Plenary Invitation, SIAM Conference on Computational Science and Engineering.}

\end{twentyshort}

%----------------------------------------------------------------------------------------
%	 new page
%----------------------------------------------------------------------------------------

\newpage % Start a new page

\makeprofileGeneral % Print the sidebar

%----------------------------------------------------------------------------------------
%	 Book chapters
%----------------------------------------------------------------------------------------
\section{Book Chapters}

\begin{twentyshort} 

	\twentyitemshort{2022}{Domino, S. P., \textit{Unstructured finite volume approaches for turbulence.}, In:  Moser, R. (ed), in Numerical Methods in Turbulence Simulation, Academic Press.}

	\twentyitemshort{2017}{Eldred, M., Ng, A., Barone, M., Domino, S. P., \textit{Multifidelity uncertainty quantification using spectral stochastic discrepancy models}, In: Ghanem R., Higdon D., Owhadi H. (eds) in Handbook of Uncertainty Quantification., Springer.}
	
  \end{twentyshort}

%----------------------------------------------------------------------------------------
%	 SNL reports
%----------------------------------------------------------------------------------------
\section{Select Sandia National Laboratories Reports}

\begin{twentyshort}
	
	\twentyitemshort{2019}{Domino, S. P., Ananthan, S., Knaus, R., Williams, A., \textit{Deploying Nalu/Kokos algorithmic infrastructure with performance benchmarking}, Sandia National Laboratories, Sandia Report SAND2017-10549R.}
	
	\twentyitemshort{2019}{Domino, S. P., Williams, A., \textit{Nalu's linear system assembly using T-Petra}, Sandia National Laboratories, Sandia Report SAND2019-0120.}
	
	\twentyitemshort{2018}{Domino, S. P., Thomas, S., Barone, M., et al., \textit{Deploying production sliding mesh capability with linear solver benchmarking}, Sandia National Laboratories, Sandia Report SAND2018-1807R.}
	
	\twentyitemshort{2015}{Domino, S. P., \textit{Sierra Low Mach Module: Nalu Theory Manual 1.0}, Sandia National Laboratories, SAND2015-3107W.}
	
	\twentyitemshort{2005}{Nicolette, V., Tieszen, S., Black, A., Domino, S. P., O'Hern, T., \textit{A turbulence model for buoyant flows based on vorticity generation}, Sandia National Laboratories, Sandia Report SAND2005-6273.}

\end{twentyshort}

%----------------------------------------------------------------------------------------
%	 CTR Briefs
%----------------------------------------------------------------------------------------
\section{Stanford Center for Turbulence (CTR) Briefs}

\begin{twentyshort}

	\twentyitemshort{2022}{Benjamin, M., Domino, S. P., Iaccarino, G., \textit{Challenges in the use of neural networks for large–eddy simulations}, Annual CTR Research Briefs.}

	\twentyitemshort{2018}{Domino, S. P., Jofre, L., Iaccarino, G., \textit{The suitability of hybrid meshes for low-Mach large-eddy simulation}, Proceedings of the CTR Summer Porogram (CTPSP). Jofre, L., Domino, S. P., Iaccarino, G., \textit{Characterization of structural uncertainty in LES of a round jet}, Annual CTR Research Briefs.}
	
	\twentyitemshort{2016}{Domino, S. P., Jofre, L., Iaccarino, G., \textit{Exploring model-form uncertainties in large-eddy simulation}, Proceedings of the CTRSP. Jofre, L., Domino, S. P., Iaccarino, G., \textit{A framework for estimating uncertainty in LES closures}, Annual CTR Research Briefs}
		
	\twentyitemshort{2014}{Domino, S. P., \textit{A comparison between low-order and higher-order low-Mach discretization approaches}, Proceedings of the CTRSP.}
	
	\twentyitemshort{2010}{Domino, S. P., \textit{Towards verification of sliding mesh algorithms for complex applications using MMS}, Proceedings of the CTRSP.}
	
	\twentyitemshort{2008}{Domino, S. P., \textit{A comparison of various equal-order interpolation methodologies using the method of manufactured solutions}, Proceedings of the CTRSP.}
	
	\twentyitemshort{2006}{Domino, S. P., \textit{Toward verification of formal time accuracy for a family of approximate projection methods using the method of manufactured solutions}, Proceedings of the CTRSP.}
	
\end{twentyshort}

%----------------------------------------------------------------------------------------
%	 Past Preparation
%----------------------------------------------------------------------------------------
\section{Past Preparation}

\begin{twenty} 

	\twentyitem{2000-2001}{Postdoctoral appointee} {Sandia National Laboratories}  {Development of a smoke transport simulation tool for cargo bay fires in support of the FAA's response to ValueJet Flight 592. This work was recognized as part of the NASA Associate Administrator's Choice Award for Outstanding Accomplishment (Glenn Research Center) and a R\&D 100 Award for the development of a multi-parameter, micro-sensor-based low false alarm fire detection system.}
  
\end{twenty}

\makeprofileGeneral % Print the sidebar  

%----------------------------------------------------------------------------------------
%	 Noteworthy Experiences
%----------------------------------------------------------------------------------------
\section{Noteworthy Experiences}

\begin{twentyshort}

	\twentyitemshort{2023 \& 2024}{Teaching Me469, Stanford Mechanical Engineering Departments Graduate class, \textit{Computational Methods for Fluid Dynamics}.}

	\twentyitemshort{2020 \& 2021}{Co-teaching Me469, Stanford Mechanical Engineering Departments Graduate class, \textit{Computational Methods for Fluid Dynamics}.}
	
	\twentyitemshort{2006-2018}{Six-time visiting scholar at Stanford's Center for Turbulence Research.}
	
	\twentyitemshort{2001-now}{Mentoring of multiple post-doctoral researchers and graduate students.}
	
	\twentyitemshort{2000-now}{Numerous internal and external peer-reviews supported including journals, DOE panels, NSF, Swiss NSF, and others.}
	
\end{twentyshort}

%----------------------------------------------------------------------------------------
%	 Misc Papers
%----------------------------------------------------------------------------------------
\section{Miscellaneous Papers}

\begin{twentyshort} 

	\twentyitemshort{2022}{Hubbard, J., Cheng, M.D., Domino, S. P., \textit{Mixing in low Reynolds number reacting impinging jets in crossflow}, J. Fluids. Engr., https://doi.org/10.1115/1.4056894}

	\twentyitemshort{2000}{Domino, S. P., Smith, P. J., \textit{State space sensitivity to a prescribed probability density function shape in coal combustion systems: Joint $\beta$-PDF versus clipped Gaussian PDF}, Proc. Combust. Inst., https://doi.org/10.1016/S0082-0784(00)80644-X.}
		
\end{twentyshort}

%----------------------------------------------------------------------------------------
%	 Conference Papers
%----------------------------------------------------------------------------------------
\section{Select Conference Papers}

\begin{twentyshort} 

	\twentyitemshort{2007}{Domino, S. P., Wagner, G., Luketa-Hanlin, A., Black, A., Sutherland, J., \textit{Verification for multi-mechanics applications}, 48$^{th}$ AAIAA/ASME/ASCE/AHS/ASC Structures, Structural Dynamics, and Materials Conference.}
	
	\twentyitemshort{2002}{Domino, S. P., Moen, C., Burns, S., Evans, G., \textit{SIERRA/Fuego: A multi-mechanics fire environment simulation tool}, 41$^{st}$ Aerospace Sciences Meeting and Exhibit.}
	
	\twentyitemshort{2002}{Domino, S. P., DesJardin, P., Suo-Antilla, J., \textit{Development of a smoke transport model to enhance the certification process for cargo bay smoke detection systems}, Fire Safety Science Proceedings of the Seventh International Symposium.}
	
\end{twentyshort}

%----------------------------------------------------------------------------------------
%	 References
%----------------------------------------------------------------------------------------
\section{References}

\begin{twentysingle} 

	\twentyitemsingle{Please contact me for a comprehensive list of references.}
	
\end{twentysingle}

\section{Review}

Dr. Stefan Domino is a computational domain specialist researcher who develops tools and techniques
to support advancement of multi-physics understanding of complex phenomena including turbulent 
fluid mechanics, heat transfer, and chemical reactions. Specific research thrusts center on high-fidelity
computational modeling and simulation approaches for fire, wave- \& wind-energy with a research thrust on sustainability (and how computational science can support advanced approaches).

\end{document}
