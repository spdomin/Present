%%%%%%%%%%%%%%%%%%%%%%%%%%%%%%%%%%%%%%%%%
% Twenty Seconds Resume/CV
% LaTeX Template
% Version 1.1 (8/1/17)
%
% This template has been downloaded from:
% http://www.LaTeXTemplates.com
%
% Original author:
% Carmine Spagnuolo (cspagnuolo@unisa.it) with major modifications by 
% Vel (vel@LaTeXTemplates.com)
%
% License:
% The MIT License (see included LICENSE file)
%
% Modified by:
% Stefan P. Domino with retention of the MIT LICENSE
%
%%%%%%%%%%%%%%%%%%%%%%%%%%%%%%%%%%%%%%%%%

%----------------------------------------------------------------------------------------
%	PACKAGES AND OTHER DOCUMENT CONFIGURATIONS
%----------------------------------------------------------------------------------------

\documentclass[letterpaper]{twentysecondcv_spd} % a4paper for A4

%----------------------------------------------------------------------------------------
%	 PERSONAL INFORMATION
%----------------------------------------------------------------------------------------

% If you don't need one or more of the below, just remove the content leaving the command, e.g. \cvnumberphone{}

\profilepic{cvMontageSmall.png} % Profile picture

\cvname{Stefan P. Domino} % Your name
\cvjobtitle{Computational Scientist} % Job title/career

\cvdate{01 January 2019} % Creation date
\cvaddress{Sandia National Laboratories \newline PO Box 5800 MS 0828 \newline Albuquerque, NM 87109-0828} % Short address/location, use \newline if more than 1 line is required
\cvnumberphone{505 284-4317} % Phone number
\cvsite{https://github.com/NaluCFD} % germane website
\cvmail{spdomin@sandia.gov} % Email address

%----------------------------------------------------------------------------------------

\begin{document}

\pagestyle{headings}
\setcounter{page}{1}
\pagenumbering{roman}

%----------------------------------------------------------------------------------------
%	 What I do
%----------------------------------------------------------------------------------------

\aboutme{ I transform people's understanding of the world by deploying high-performing computational 
fluid dynamics tools. While some draw analogy to such computational tools as the rasp in Michelangelo's hand, 
I choose to view the partnership analogous to that of the luthier and the violinist in that both are required 
to make new that which was formerly unheard.} 

%----------------------------------------------------------------------------------------
%	 SKILLS
%----------------------------------------------------------------------------------------

% Skill bar section, each skill must have a value between 0 an 6 (float)
\skills{{Passionate/5.45}, {Driven/5.5}, {Outgoing/4.0}, {Software Development/5.0}, {NGP/4.8}, {Turbulence/5.1}, {low-Mach Fluids/5.4}}

%------------------------------------------------

%----------------------------------------------------------------------------------------

\makeprofile % Print the sidebar

%----------------------------------------------------------------------------------------
%	 INTERESTS
%----------------------------------------------------------------------------------------

\section{Interests}

My professional interest rests in the development and deployment of computational fluid dynamics 
tools, which can be used in a wide range of applications, to facilitate a transformative understanding 
of otherwise intractable physical phenomena. By using these tools, in partnership with theory and
experiments, a window within complex coupled processes can be unsealed. Studying low-Mach multi-physics 
applications that include turbulence, variable density effects, buoyancy, and chemical reactions often times
reveals extraordinarily complex fluids, thermal, and species structures thereby allowing that which is
generally unseen to be fully appreciated. Fostering partnerships within a diverse and high-performing 
team required to solve grand-challenge problems provides motivation to work in the complex field of CFD.

%----------------------------------------------------------------------------------------
%	 EDUCATION
%----------------------------------------------------------------------------------------

\section{Education}

\begin{twenty} % Environment for a list with descriptions
	%\twentyitem{<dates>}{<title>}{<location>}{<description>}
	\twentyitem{1999}{Doctor of Philosophy} {\normalfont Chemical Engineering, University of Utah} {Research advanced
modeling and simulation techniques in an effort to more accurately predict the oxides of Nitrogen (NOx) in pulverized coal
furnaces}
\end{twenty}

%----------------------------------------------------------------------------------------
%	 PUBLICATIONS
%----------------------------------------------------------------------------------------

\section{Peer-reviewed Publications}

\begin{twentyshort} % Environment for a short list with no descriptions
	\twentyitemshort{2019}{An assessment of atypical mesh topologies for low-Mach large-eddy simulation; 
Domino, Sakievich, and Barone; Comput.\& Fluids, 179 (30).}
	\twentyitemshort{2018}{Design-order, non-conformal low-Mach fluid algorithms using a hybrid CVFEM/DG 
approach; Domino; JCP, 359 (15).}
	\twentyitemshort{2018}{A framework for characterizing structural uncertainty in large-eddy simulation 
closures; Jofre, Domino, and Iaccarino; Flow Turb.\& Comb., 100 (2).}
	\twentyitemshort{2018}{Uncertainty quantification in LES of channel flow; Safta, Blaylock, 
Templeton, Domino, and Najm; IJNMF., 83 (4).}
        \twentyitemshort{2017}{Multifidelity uncertainty quantification using spectral stochastic discrepancy 
models; Eldred, Ng, Barone, and Domino; In: Ghanem R., Higdon D., Owhadi H. (eds) Handbook of Uncertainty 
Quantification.}
\end{twentyshort}

%----------------------------------------------------------------------------------------
%	 AWARDS
%----------------------------------------------------------------------------------------

\section{Awards}

\begin{twentyshort} % Environment for a short list with no descriptions
	\twentyitemshort{2017}{Sheldon R. Tieszen Engineering Sciences Award for extensive technical excellence.}
	\twentyitemshort{2000-2018}{Several Sandia National Laboratories institutional achievement awards.}
\end{twentyshort}

%----------------------------------------------------------------------------------------
%	 EXPERIENCE
%----------------------------------------------------------------------------------------

\section{Experience}

\begin{twenty} % Environment for a list with descriptions
	\twentyitem{2005-now}{Principal Member of the Technical Staff} {Sandia National Laboratories} {Principal Investigator of an
ASC VVUQ project centered on assessing structural uncertainty for LES; orginator of the open-source 
Nalu code base; member of the ExaWind team (Office of Science ECP, multi-laboratory, multi-institutional project) that is
driving wind farm predictions on next generation platforms.}
	\twentyitem{2001-2005}{Senior Member of the Technical Staff} {Sandia National Laboratories}  {Principal Investigator
 and lead developer for the generally unstructured, massively parallel Sierra low-Mach module Fuego code base. This fire mechanics
simulation tool supports NNSA's mission of Science-based Stockpile Stewardship.}
	\twentyitem{2000-2001}{Postdoctoral appointee} {Sandia National Laboratories} {Development of smoke transport simulation
tools for use in cargo bay fires.}
	\twentyitem{1996-2000}{Research Assistant} {University of Utah} {Graduate student within the chemical engineering department
 focused on computational approaches for improved NOx prediction.}
\end{twenty}

%----------------------------------------------------------------------------------------
%	 SECOND PAGE EXAMPLE
%----------------------------------------------------------------------------------------

\newpage % Start a new page

%----------------------------------------------------------------------------------------
%	 talk about why I do what I do
%----------------------------------------------------------------------------------------

\whyidoit{ The ability to explore multi-physics applications from a foundational modeling
and simulation perspective is critical to future scientific advances. This high-level motivation
has driven my desire to work within the intersection of physics elucidation, numerical 
methods development, and code development. More recently, the ability to deploy advanced uncertainty
quantification (UQ) techniques to drive physics understanding, which may include structural uncertainty methods, 
machine learning approaches, etc., has transformed the former research paradigm. } 

%----------------------------------------------------------------------------------------
%	 favorite things
%----------------------------------------------------------------------------------------

% favorite things bar section, value between 0 an 6 (float)
\favthings{{Ocean/5.9}, {Science/5.7}, {Pursuit of Knowledge/5.6}, {CFD/5.8}, {Snow/5.2}, {Mountains/5.8}, {Family/6.0}, {Replication of Past Work/2.0}}

\makeprofileSecond % Print the sidebar

%----------------------------------------------------------------------------------------
%	 Goals
%----------------------------------------------------------------------------------------

\section{Goals}

My primary career goals are to extend the state-of-the art in computational fluid dynamics 
to facilitate the advanced deployment and acceptance of this tool to support a wide and 
novel range of multi-physics applications. Mentoring, teaching, and motivating the next 
generation of computational scientists also represents a core passion.

\section{Select Publications}

\begin{twentyshort} % Environment for a short list with no descriptions
	\twentyitemshort{2018}{The suitability of hybrid meshes for low-Mach large-eddy simulation; 
Domino, Jofre, and Iaccarino; Proceedings of the 2018 Center for Turbulence Research Summer Program; Stanford University.}
	\twentyitemshort{2018}{Characterization of structural uncertainty in large-eddy simulation of a circular jet; 
Domino, Jofre, and Iaccarino; Center for Turbulence Research Annual Research briefs; Stanford University.}
	\twentyitemshort{2016}{Exploring model-form uncertainties in large-eddy simulations; 
Domino, Jofre, and Iaccarino; Proceedings of the 2016 Center for Turbulence Research Summer Program; Stanford University.}
	\twentyitemshort{2014}{A comparison between low-order and higher-order low-Mach discretization approaches; 
Domino; Proceedings of the 2014 Center for Turbulence Research Summer Program; Stanford University.}
	\twentyitemshort{2013}{A reflection of recent ASC milestones in support of the abnormal/thermal enviornemnt; 
Domino; Sandia National Laboratories Technical Report, SAND2013-3927P.}
	\twentyitemshort{2010}{Towards verification of sliding mesh algorithms for complex applications using MMS; 
Domino; Proceedings of the 2010 Center for Turbulence Research Summer Program; Stanford University.}
\end{twentyshort}

%----------------------------------------------------------------------------------------
%	 OTHER INFORMATION
%----------------------------------------------------------------------------------------

\section{Noteworthy Experiences}

\begin{twentyshort} % Environment for a short list with no descriptions
	\twentyitemshort{2018}{Guest lecturer for Stanford's ME469 Mechanical Engineering Graduate Computational Fluid Mechanics class.}
        \twentyitemshort{2006-2018}{Six-time visiting scholar at Stanford's Center for Turbulence Research.}
	\twentyitemshort{2000-2018}{Numerous internal and exteral peer-review processes supported, e.g., journals, DOE panels, etc.).}
\end{twentyshort}

\section{References}

\begin{twentysingle} % Environment for a short list with no descriptions
	\twentyitemsingle{Please contact me for a comprehensive list of references.}
\end{twentysingle}

\section{Review}

Dr. Stefan Domino is a computational domain specialist researcher who develops tools and techniques
to support advancement of multi-physics understanding of complex phenomena including turbulent 
fluid mechanics, heat transfer, and chemical reactions. 

\end{document}
